Intro to what this section will present

\subsection{Materials Used}
List of materials used with relevant documentation

\begin{itemize}
  \item Arduino Uno board
  \item RFM22 ISM Tranceiver module
  \item 5kg Load Cell
  \item HX711 amplifier module
  \item Breadboard
  \item Jumper wires
  \item 3D-printed cup base
\end{itemize}

We could also include a list of materials used for calibration (standard water cup and scale)

\subsection{Load Cell}
Section to explain how the load cell works and why we need the amplifier%%%%%%%%%%%%%

\subsubsection{Taring and Calibration}
Before the load cell can produce meaningful measurements, it must be properly zeroed and scaled. This is achieved through two essential steps: \textit{taring} and \textit{calibration}.

\paragraph{Taring} is the process of removing any initial mechanical offset or weight from the sensor. When the system is powered on or reset, a \texttt{tare()} function is executed, which sets the current load cell reading to zero — assuming that no weight is applied. This is crucial for eliminating small offsets caused by the base structure or assembly.

\paragraph{Calibration} involves placing a known weight on the coaster and recording the raw value from the HX711 output. This value is then used to compute a \textit{calibration factor}, which scales future measurements to actual weight units.


explain how the user doew the taring with the button


\subsection{3D printed cup base}
explain the logic of the design and add figures

\subsection{Step by Step Set-up}
explain step by step set up 