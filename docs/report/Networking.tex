The HydroTracker system utilizes a custom wireless communication network built on \textit{Arduino microcontrollers} and \textit{RFM22 ISM-band transceivers}. 
Each coaster-shaped transmitter node, associated with a water glass, continuously monitors weight data using an \textit{HX711 load cell}. 
This data is processed locally and transmitted wirelessly to a central receiver node. 
The receiver then relays this information to a host computer for real-time visualization.

In this section, we will describe the code implementation for both the receiver and transmitter nodes, and explain the wireless communication protocol used in the system.

\subsection{Wireless Communication Protocol}

The system uses the RF22Router library, which provides a basic routing layer over the RFM22 radio transceivers. 
Each transmitter node and the receiver are assigned unique addresses, enabling reliable routing and identification of messages. 
Communication operates on the \textit{444.0 MHz} frequency using \textit{Gaussian Frequency Shift Keying (GFSK)} modulation with a data rate of \textit{125 kbps}.

To manage potential transmission collisions, the system implements a simplified \textit{ALOHA protocol} in the transmitter code:
\begin{itemize}
    \item Each transmitter sends a data packet using \texttt{sendtoWait()}, which waits for an acknowledgment from the receiver.
    \item If the acknowledgment is not received (indicating a failed or collided transmission), the transmitter waits for a \textit{random backoff delay} between 200 ms and 3000 ms before retrying.
    \item This randomized retransmission reduces the chance of repeated collisions when multiple nodes attempt to send data at the same time.
\end{itemize}

\begin{lstlisting}[language=C++, caption={ALOHA protocol implementation in the transmitter}]
bool successful_packet = false;
while (!successful_packet){

    if (rf22.sendtoWait(data_send, msg_len, DESTINATION_ADDRESS) != RF22_ROUTER_ERROR_NONE)
    {
        Serial.print("Sending: ");
        Serial.println(data_read);
        long randNumber = random(200, 3000)
        delay(randNumber);
    }
    else
    {
        successful_packet = true;
        Serial.println("Packet sent successfully!");
    }
}
\end{lstlisting}

This approach is particularly suited for low-traffic, sporadic communication systems, such as HydroTracker, where nodes send updates only when state changes are detected. 

\subsection{Receiver}
The receiver node is responsible for collecting data from multiple transmitters and relaying it to the host computer.
Its primary tasks are:

\begin{itemize}
    \item Initialize the RF22 radio module.
    \item Define known routes to each transmitter node.
    \item Continuously receive incoming data packets.
    \item Format the received data as \texttt{Node <node\_id> - State: <state>}.
    \item Forward the formatted data to the host computer via serial communication for further processing and display.
\end{itemize}

\subsection{Transmitters}
Each transmitter node is responsible for monitoring the weight of an individual glass and wirelessly reporting its status to the receiver node.
The transmitter code performs the following key functions:

\begin{enumerate}
    \item \textbf{Load cell initialization:}{
    Each transmitter uses an \textit{HX711} load cell amplifier to measure the weight of the glass. 
    The initialization of the load cell is performed during the \texttt{setup()} function.
    \begin{itemize}
        \item The data and clock pins of the HX711 are defined and used to create a \texttt{HX711\_ADC} object named \texttt{LoadCell}.
        \item The \texttt{LoadCell.begin()} function initializes the connection to the sensor.
        \item The \texttt{LoadCell.start(2000, tare)} function starts the sensor with a 2-second warm-up period and performs a tare (zeroing) operation.
        \item If initialization is successful, a calibration factor is applied using \texttt{LoadCell.setCalFactor()}. This factor is determined empirically through a calibration process and converts raw readings into meaningful weight units.
    \end{itemize}


    \begin{lstlisting}[language=C++, caption={Load cell initialization}]
    #include <HX711_ADC.h>

    const int HX711_DOUT = 4; // HX711 data pin
    const int HX711_SCK = 5;  // HX711 clock pin
    HX711_ADC LoadCell(HX711_DOUT, HX711_SCK);

    void setup() {
        LoadCell.begin();
        bool tare = true;
        LoadCell.start(2000, tare);

        if (LoadCell.getTareTimeoutFlag() || LoadCell.getSignalTimeoutFlag())
        {
            Serial.println("HX711 not found. Check wiring and pin designations.");
            while (1);
        }
        else
        {
            float calibrationFactor = 448.31; // Calibration Factor
            LoadCell.setCalFactor(calibrationFactor);
        }
    }
    \end{lstlisting}}

    \item \textbf{Weight monitoring and State classification:}{
    The function \texttt{getCurrentState()} is responsible for determining the current state of the transmitter node based on the measured weight. 
    It uses predefined threshold values to classify the state of the glass as either \texttt{NOTEXIST}, \texttt{EMPTY}, or \texttt{FULL}.

    Before classification, the function ensures that the weight reading is stable by repeatedly sampling the load cell until the weight change falls within a ±1g margin.
    This is necessary because, after a change in weight, the load cell may take a few seconds to stabilize.

    The thresholds are defined based on the fact that the load cell is tared with an empty glass placed on it. 
    This implies the following:
    \begin{itemize}
        \item \textbf{Values around 0g:} The glass is empty.
        \item \textbf{Negative values:} The glass does not exist.
        \item \textbf{Positive values:} The glass contains some amount of water.
    \end{itemize}

    \begin{lstlisting}[language=C++, caption={Weight monitoring and state classification}]
    const float NOTEXIST_THRESHOLD = -20.0; // Threshold for no existence
    const float EMPTY_THRESHOLD = 50.0;    // Threshold for empty

    state getCurrentState(){

        LoadCell.update();
        float currentWeight = LoadCell.getData();

        while (1)
        {
            delay(1000);
            LoadCell.update();
            float weight = LoadCell.getData();

            if (weight < currentWeight + 1.0 && weight > currentWeight - 1.0)
            {
                currentWeight = weight;
                break; // Exit the loop if weight is stable
            }

            currentWeight = weight;
        }

        if (currentWeight < NOTEXIST_THRESHOLD) return NOTEXIST;
        else if (currentWeight < EMPTY_THRESHOLD) return EMPTY;
        else if (currentWeight >= EMPTY_THRESHOLD) return NOTEMPTY;
    }
    \end{lstlisting}}

    \item \textbf{Tare function:}{
    The user can manually tare the load cell by pressing a button connected to a digital pin.
    The tare function is triggered when the button is pressed, and it sets the current weight reading to zero.
    This is useful for recalibrating the load cell when a new glass is placed on it.
    
    \begin{lstlisting}[language=C++, caption={Tare function}]
    if (digitalRead(TARE_BUTTON_PIN) == HIGH)
    {
        Serial.println("Tare button pressed. Taring...");
        digitalWrite(LED_PIN, HIGH);
        LoadCell.tare();
        digitalWrite(LED_PIN, LOW);
    }
    \end{lstlisting}}

    \item \textbf{Data transmission:}{
    The packet transmission is handled by the \texttt{sendPacket()} function. 
    This function formats the current state as a string (e.g., \texttt{"EMPTY"}, \texttt{NOTEMPTY"}, \texttt{"NOTEXIST"}) and attempts to transmit it to the receiver node using the \textit{ALOHA protocol}.

    To optimize network usage and conserve energy, the transmitter only sends a packet when a change in the glass state is detected.
    To achieve this, in the \texttt{loop()} function, the following steps are performed:
    \begin{itemize}
        \item The transmitter maintains a record of the \texttt{previousState}.
        \item On each loop iteration, the current state is determined using \texttt{getCurrentState()}.
        \item If the state is unchanged, no packet is sent.
        \item If the state changes, the new state is sent using \texttt{sendPacket()} and \texttt{previousState} is updated accordingly.
    \end{itemize}

    \begin{lstlisting}[language=C++, caption={Packet transmission on state change}]
    state currentState = getCurrentState();

    if (currentState != previousState)
    {
        sendPacket(currentState);
        previousState = currentState;
    }
    else
    {
        Serial.print("State unchanged: ");
        Serial.println(stateToString(currentState));
    }
    \end{lstlisting}}
\end{enumerate}